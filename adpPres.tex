\documentclass[mathserif]{beamer}
%\documentclass[mathserif,handout]{beamer}
%\documentclass[dvips]{beamer}
\definecolor{Beige}{rgb}{0.96,0.96,0.86}
\definecolor{Yellow}{rgb}{1.,0.84,0.8}
\definecolor{Gold}{rgb}{1.,0.84,0.}
\definecolor{RedA}{hsb}{0.9,0.3,0.7}
\definecolor{RedB}{hsb}{0.9,0.3,1}
\definecolor{LightGray}{gray}{0.85}
\definecolor{Blue}{rgb}{0.,0.,1.}
\definecolor{Pink}{rgb}{0.9,0.75,0.8}
\definecolor{DarkGreen}{rgb}{1.0,1.0,1.0}
\definecolor{DarkBlue}{rgb}{0.,0.5,1.0}
%\definecolor{Pink}{rgb}{1.,0.75,0.8}
\definecolor{LightCyan}{rgb}{0.88,1.,1.}
%\definecolor{LightYellow}{rgb}{0.95,1.0,0.8}
\definecolor{red1}{rgb}{0.95,0.,0.}
%\definecolor{mycol1}{rgb}{0.9,0.2,0.9}
%\definecolor{mycol1}{rgb}{0.5,0.8,0.4}
\definecolor{mycol1}{rgb}{0.5,0.6,0.8}
\definecolor{mycol2}{rgb}{0.6,0.8,0.6}
\definecolor{LightYellow}{rgb}{0.95,1.0,0.8}

\mode<presentation>
{
%\usepackage{beamerthemeCambridgeUS, fancybox}
\usepackage{fancybox}
\usepackage{graphicx}
\usepackage{subfigure}
\usepackage[english]{babel}
\usepackage{multirow}
\usepackage{verbatim}
\usepackage{verbatim,epsfig,graphics,amssymb,amsmath,subfigure}
\usepackage{helvet}
\usepackage{graphicx}
\usepackage{wrapfig}
\setbeamercovered{transparent}
}
%\usepackage[T1]{fontenc}
%\usepackage[adobe-utopia]{mathdesign}
%\usepackage{fouriernc}
\usepackage[bitstream-charter]{mathdesign}
\usepackage[T1]{fontenc}



\usefonttheme{serif}

\usetheme{Boadilla} 

\setbeamertemplate{navigation symbols}{}
\setbeamersize{text margin left=3mm, text margin right=3mm}
\newcommand{\B}[1]{\boldsymbol{#1}}
\usepackage[all]{xy}
\usepackage[utf8]{inputenc}
%\usepackage{pdfpages}
\renewcommand{\footnotesize}{\scriptsize}

\title[Seed-grant proposal]
{Multimodal classification of birds\\
\small{Seed-grant proposal}}

%\author[Paddy]{R. Padmanabhan (Paddy)}
\author[ADP]{Arnav Bhavsar\\
		Dileep A. D.\\
		Padmanabhan Rajan
}

\institute[IIT Mandi] {
Multimedia Analytics and Systems Lab\\
School of Computing and Electrical Engineering\\
%Indian Institute of Technology Mandi \\
\includegraphics[width=4cm,height=2cm]{figures/mas_logo.pdf}
\includegraphics[width=3cm,height=2cm]{figures/iitmandi-logo.pdf}
}
%\date[November 11th, 2009]
{}

%%%%%%%%%%%%%%%%%%%%%%%%%%%%%%%%%%%%%%%%%%%%%%%%%%%%%%%%%%%%%%%%%%%%%%%%%%%%%%%%%%%%%%%%%%%%%
\begin{document}
\normalfont

%%%%%%%%%%%%%%%%%%%%%%%%%%%%%%%%%%%%%%%%%%%%%%%%%%%%%%%%%%%%%%%%%%%%%%%%%%%%%%%%
\begin{frame}
        \titlepage
\end{frame}

%\logo{\includegraphics[height=0.8cm]{figures/iitmandi-logo.pdf}\vspace{220pt}}
\logo{\includegraphics[height=0.8cm]{figures/mas_logo.pdf}\vspace{220pt}}

\begin{frame}
\frametitle{Overview}
\begin{wrapfigure}{r}{0.5\textwidth}
\centering
%  \begin{center}
    \includegraphics[width=0.48\textwidth]{figures/parakeet.jpg}
%  \end{center}
  \caption{\small{Slaty-headed parakeet. Pic by PPJ.}}
\end{wrapfigure}
The objective\\
The acoustics\\
The image/video\\
The machine learning\\
The budget and other details\\
\end{frame}

\begin{frame}
\frametitle{The objective}
\begin{itemize}
\item<2-> Develop algorithms for automatic analysis of avain biodiversity
\item<3-> Combine information from acoustic and visual data streams
\item<4-> Sensors: microphones, cameras
\item<5-> Apply signal processing and machine-learning techniques to collected
data
\item<6-> Tasks: Species identification, species detection
\end{itemize}
\end{frame}

\begin{frame}
\frametitle{The motivation}
\begin{itemize}
\item<2-> Birds provide crucial ecosystem services: pollination, seed dispersal,
insectivory
\item<3-> Avian diversity: good indicator of ecosystem health in a local area
\item<4-> Automatic and semi-automatic sensing devices can be utilised
\item<5-> Large volume of data captured by these devices
\item<6-> Algorithms to analyse this data would be useful to ecologists
\item<7-> Our campus location in the lower Himalays: sensitive ecosystem
\item<8-> Proposed system can be used for long-term ecological monitoring
\end{itemize}
\end{frame}

\begin{frame}
\frametitle{The challenges}
\begin{itemize}
\item<2-> Challenges at various levels
\item<3-> Acoustics:
	\begin{itemize}
	\item<4-> Complex acoustic environment where recordings are made
	\item<5-> Overlapping vocalisations, intra-species call variability
	\item<7-> Background sounds (other animals, human-made sounds, river etc.)
	\end{itemize}
\item<8-> Image/video 
	\begin{itemize}
	\item<9-> Complex visual environment, visual background clutter
	\item<10-> Overlapping inter-class visual appearances, local variations  
	\item<11-> Intra-class variations: Robustness to changes in pose, motion, light conditions.
	\end{itemize}
% \item<12-> Fusion of modalities
% 	\begin{itemize}
% 	\item<13-> Curating of feature vectors and subsystem decisions 
% 	\item<14-> Complementary representations from different modalities
% 	\item<15-> Correlating sound and videos
% 	\end{itemize}
\item<15-> Machine-learning
	\begin{itemize}
	\item<16-> Classification using Fixed-length representations and varying length representations
	\item<17-> Dynamic kernels for bird data from different modalities
	\item<18-> Fusion of modalities
	\begin{itemize}
	    \item<19-> Combining the representations from aoustic and image/video modes
	    \item<20-> Combining the decicions from the classifiers for the representations from aoustic and image/video modes
	\end{itemize}
	\item<21-> Bird indexing and retrieval
	\end{itemize}
\end{itemize}
\end{frame}


\begin{frame}
\frametitle{The acoustics}
\vspace{-1cm}
Automatic analysis of birdcalls: fairly active research area\\
\vspace{0.3cm}
\pause
Data acquisition: rugged, field-deployable recorders \\
eg.~Song Meter SM3 recorder from Wildlife Acoustics Inc, USA.
\begin{wrapfigure}{r}{0.3\textwidth}
	\centering
	\only<2>{\includegraphics[width=0.3\textwidth]{figures/songMeter.png}}
	%\caption{\small{Song Meter SM3 recorder}}
\end{wrapfigure}
\end{frame}


\begin{frame}
\frametitle{The acoustics (cont'd)}
\begin{itemize}
\item<2-> Processing of human speech: techniques can be adapted for birdcalls
\item<3-> Production mechanims are different, but have similarities (eg. formant
structure)
\item<4-> Existing techniques include:
\begin{itemize}
	\item spectral representations \footnote{
	H. Tyagi et. al., ``Automatic identification of bird calls using spectral 
	ensemble average voice prints'', Proc. EUSIPCO, 2006},
	\item Mel frequency cepstra \footnote{M. Graciarena et. al., 
	``Acoustic front-end optimization for bird species recognition'', 
	Proc. ICASSP, 2010}, 
	\item hidden Markov models \footnote{M. Graciarena et.al., 
	``Bird species recognition combining acoustic and sequence modeling'', 
	Proc. ICASSP, 2011}, 
	\item sparse representations \footnote{L. N. Tan et. al. 
	``Evaluation of a sparse representation-based classifier for bird phrase 
	classification", Proc. Interspeech, 2012}
\end{itemize}
\end{itemize}
\end{frame}


\begin{frame}
\frametitle{The acoustics (cont'd)}
\begin{itemize}
\item<2-> Research focus: \textcolor{blue}{subspace representations} 
\item<3-> A recording can be represented as a fixed-length vector $\mathbf{x}$ 
\item<4-> Can be used for various applications, for eg. removing background
sounds before classification:
\begin{itemize}
	\item<5-> Project $\mathbf{x}$ into a subspace of background sounds, and remove this
	component from $\mathbf{x}$
\end{itemize}
\item <6-> Fixed-length representations: utilised in kernel functions for
support vector machines
\end{itemize}
\end{frame}


\begin{frame}
\frametitle{The visual: Research areas}
\begin{itemize}
\item<2-> Fine-grained classification (of birds): Relatively recent research area ($\geq$ 2010)
\item<3-> Detection, segmentation and tracking of birds (relatively unexplored): \\Adapting general object detection, segmentation and tracking methods.
\item<4-> Learning visual guidance: Inverse problem to classification \\$\Rightarrow$ Given the classes, find the discriminative features
\item<5-> Sound source localization (active area in general domains): \\Challenges for birds: less visual motion, background sounds
\end{itemize}
\end{frame}


\begin{frame}
\footnotesize
\frametitle{The visual: Existing work}
\begin{itemize}
	\item<2-> Fine-grained classification 
	\begin{itemize}
	\footnotesize
	\item H. Tyagi et. al., ``Automatic identification of bird calls using spectral 
	ensemble average voice prints'', Proc. EUSIPCO, 2006,
	\item M. Graciarena et. al., ``Acoustic front-end optimization for bird species recognition'', 
	Proc. ICASSP, 2010, 
	\item M. Graciarena et.al., ``Bird species recognition combining acoustic and sequence modeling'', 
	Proc. ICASSP, 2011, 
	\end{itemize}	
	\item<3-> Visual guidance: 
	\item<4-> Detection, Segmentation
	\begin{itemize}
	\footnotesize
	\item H. Tyagi et. al., ``Automatic identification of bird calls using spectral 
	ensemble average voice prints'', Proc. EUSIPCO, 2006,
	\item M. Graciarena et. al., ``Acoustic front-end optimization for bird species recognition'', 
	Proc. ICASSP, 2010, 
	\item M. Graciarena et.al., ``Bird species recognition combining acoustic and sequence modeling'', 
	Proc. ICASSP, 2011, 
	\end{itemize}
\end{itemize}
\end{frame}

\begin{frame}
\frametitle{The visual: Possible directions}
\begin{itemize}
\item<2-> Features and frameworks: 
\begin{itemize}
\item Patch-based features 
\item Body-part features: Appearance and Geometric relationships
\item Feature learning: Deep neural networks, Discriminative features
\end{itemize}
\item<3-> Frameworks:
\begin{itemize}
\item Sparse representation
\item Markov Random Fields
\item Hierarchical classification 
\end{itemize}
\item<4-> Systems:
\begin{itemize}
\item Dataset collection
\item Audio-video systems for monitoring
\item On-board algorithms: Detection, tracking
\end{itemize}
\end{itemize}
\end{frame}


\begin{frame}
\frametitle{The machine learning: Tasks}
\begin{itemize}
\item<2-> Bird call identification using fixed length and varying length accoustic features 
\item<3-> Bird classification from images and videos
\item<4-> Bird call indexing and retreval
\item<5-> Bird image and video indexing and retrieval
\item<6-> Combining different modalities for classification, indexing and retrieval tasks
\end{itemize}
\end{frame}

\begin{frame}
\frametitle{The machine learning: Bird classification}
\begin{itemize}
\item<2-> Classification of birds using support vector machines (SVMs) from bird calls and bird images \& videos
\item<3-> The representations for bird call are either fixed length representation or varying length representation
\item<4-> Varying length representation are either sets of local feature vectors or sequences of local feature vectors
\item<5-> Dynamic kernel based SVMs for varying length representation
\end{itemize}
\end{frame}

\begin{frame}
\frametitle{The machine learning: Bird classification (cont'd)}
\begin{itemize}
\item<2-> Some of the dynamic kernels are:
\begin{itemize}
	\item GMM-based intermediate matching kernel \footnote{
	A. D. Dileep and C. Chandra Sekhar, "GMM-Based intermediate matching kernel for Classification of Varying Length Patterns of Long Duration Speech Using Support Vector Machines," in IEEE Transactions on Neural Networks and Learning Systems, vol. 25, no. 8, pp. 1421-1432, Aug. 2014},
	\item HMM-based intermediate matching kernel \footnote{A. D. Dileep and C. Chandra Sekhar, “HMM based intermediate matching kernel for classification of sequential patterns of speech using support vector machines,” IEEE Transactions on Audio, Speech and Language Processing , vol. 21, no. 12, pp. 2570-2582, Dec. 2013}, 
	\item Histogram intersection kernel \footnote{J. C. van Gemert, Cor J. Veenman, A. W. M. Smeulders, and Jan-Mark Geusebroek, “Visual word ambiguity,” IEEE Transactions on Pattern Analysis and Machine Intelligence, vol. 32, no. 17, pp. 1271–1283, July 2010}, 
	\item Spacial pyramid match kernel \footnote{S. Lazebnik, C. Schmid, and J. Ponce, “Beyond bags of features:Spatial pyramid matching for recognizing natural scene categories,” in Proceedings of the 2006 IEEE Computer Society Conference on Computer Vision and Pattern Recognition (CVPR 2006), New York, USA, June 2006, pp. 2169–2178}
\end{itemize}
\end{itemize}
\end{frame}

\begin{frame}
\frametitle{The machine learning: Bird indexing and retrieval}
\begin{itemize}
\item<2-> Maching and retrieval of birds using bird calls and bird images \& videos
\begin{itemize}
      \item Query-by-example (QBE) based retrieval\footnote{
	A. Marakakis, N. Galatsanos, A. Likas, and A. Stafylopatis, “Probabilistic relevance feedback approach for content-based image retrieval based on Gaussian mixture models, “ IET Image Processing, vol. 3, no. 1, pp. 10-25, February 2009}
      \item Query-by-semantics (QBS) based retrieval\footnote{
	G. Carneiro, A. B. Chan, P.J. Moreno, and N. Vasconcelos, “Supervised learning of semantic classes for image annotation and retrieval,” IEEE Transactions on Pattern Analysis and Machine Intelligence, vol. 29, no. 3, pp. 394-410, March 2007}
      \item Query-by-semantic example (QBSE) based retrieval\footnote{
	N. Rasiwasia, P. J. Moreno, and N. Vasconcelos, “Bridging the gap: Query by semantic example,” IEEE Transactions on Multimedia, vol. 9, no. 5, pp. 923-938, August 2009}
\end{itemize}
\item<3-> Matching and retrieval of birds using kernel methods\footnote{
	T. Veena, “Image classification, matching and annotation using kernel methods for content based image retrieval for scene images,” Ph.D. Thesis, Dept. of Computer Science and Eng., IIT Madras, Chennai, India, 2005.}   
\end{itemize}
\end{frame}

\begin{frame}
\frametitle{The machine learning: Multimodal classification and retrieval}
\begin{itemize}
\item<2-> Classfication and retrieval of birds by combining the ques from bird calls and bird images \& videos
\begin{itemize}
      \item Early fusion: Combining the acoustic, image and video features
      \item Late fusion: Combining the decisions from the different classifiers built for bird calls, bird images and bird videos
 \end{itemize}
\item<3-> Feature selection and combining using multiple kernel learning
\end{itemize}
\end{frame}

\begin{frame}
\frametitle{Budget and other details}

\begin{table}[th]
\centering
\caption{Projected expenses in lakhs INR.}
\begin{tabular}{|c|c|c|c|c|}
\hline
Items & Year 1 & Year 2 & Year 3 & Total\\
\hline
High-end computers (2) & 3.0 & 3.0 & 0 & 6.0\\
Imaging and audio equipment & 5.0 & 3.0 & 0 & 8.0 \\
Desktop computers (6) & 5.0 & 0 & 0 & 5.0  \\
Contingency & 0.5 & 0.5 & 1.0 & 2.0 \\
Travel & 0.5 & 0.5 & 1.0 & 2.0\\
\hline
Overall & 15.0 & 8.0 & 2.0 & \textbf{23.0} \\
\hline
\end{tabular}
\label{tab:funding}
\end{table}
\end{frame}


\begin{frame}
\frametitle{Future plans: Further funding}
\begin{itemize}
\item \textbf{Proposal to SERB (ready for submission)}:
\item \textbf{Proposal planned:} Camera and acoustic sensor networks for a local area (IIT Mandi campus)
\end{itemize}
\end{frame}


\begin{frame}
\frametitle{}
\Large{Thank you for your attention.}
\end{frame}


\end{document}

